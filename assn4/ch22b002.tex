\documentclass{article}
\usepackage{amsmath}
\usepackage{physics}
\usepackage{cite}
\usepackage{graphicx}
\usepackage{hyperref}
\usepackage{fancyhdr}
\usepackage[symbol]{footmisc}
\usepackage{xcolor}
\usepackage{fancyhdr}
\usepackage{soul}
\pagestyle{fancy}
\pagenumbering{arabic}
\renewcommand{\headrulewidth}{0.4pt}
\renewcommand{\footrulewidth}{0.4pt}

\lhead{ID2090}
\rhead{Assignment-4}


\begin{document}


\begin{titlepage}
    \centering
    \vfil
    {\bfseries\Large
        \vspace*{\stretch{1}}
        \huge{Assignment-4}\\
        \vskip1cm
        ID2090\\
        \vskip2cm
        Ganesh B\\
        \vskip0.3cm
        CH22B002\\
        \vskip0.3cm
        GitHub: {\color{blue}{{\href{https://github.com/gnshb}{{gnshb}}}}}\\
        \vspace*{\stretch{1}}   
    }    
    \vfil
\end{titlepage}
\newpage

\section{Vandermonde Determinant}

\subsection{The Vandermonde Matrix}

In linear algebra, a Vandermonde matrix, named after Alexandre-Théophile Vandermonde, is a ${(m+1)\times (n+1)}$ matrix with the terms of a geometric progression in each row: 

$$
V=V\left(x_0, x_1, \cdots, x_m\right)=\left[\begin{array}{ccccc}
1 & x_0 & x_0^2 & \ldots & x_0^n \\
1 & x_1 & x_1^2 & \ldots & x_1^n \\
1 & x_2 & x_2^2 & \ldots & x_2^n \\
\vdots & \vdots & \vdots & \ddots & \vdots \\
1 & x_m & x_m^2 & \ldots & x_m^n
\end{array}\right]_{(m+1)\times(n+1)}
$$

The Vandermonde matrix and its determinant\footnote[4]{i.e. when $m=n$ or $V$ is square} have several important applications. Let us first look at its determinant.

\subsection{Determinant}

The determinant is surprisingly very simple to calculate. It can be put very compactly as a product:

$$\boxed{\left| V \right|=\prod_{0\le i<j\le n}\left(x_j-x_i\right)}$$

There are many ways to prove this. A couple of them are listed here\cite{citation1}.
\subsection{Applications}

There are various instances where this matrix pops up. In polynomial interpolation, since inverting the Vandermonde matrix (given $n=m$ ) allows expressing the coefficients of the polynomial in terms of the $x_i$ and the values $p\left(x_i\right)$ of the polynomial at those points. In other words, given $n+1$ distinct nodes with coordinates $\left(x_i, p\left(x_i\right)\right)$, there exists a unique polynomial $p(x)=a_0+a_1 x+a_2 x^2+\cdots+a_n x^n$ of degree $n$ which interpolates them and its coefficients can be computed by solving $a=V^{-1} f$, where $a$ is the column coefficient matrix. The discrete Fourier transform matrix is a very special example of a Vandermonde matrix\cite{citation2}.
\vskip0.5cm
The rows also form orthogonal bases, which have many nice applications in number theory and linear algebra.



\newpage
\bibliography{mybib}{}
\bibliographystyle{ieeetr}

\end{document}